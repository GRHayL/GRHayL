\documentclass{article}

% Use the Cactus ThornGuide style file
% (Automatically used from Cactus distribution, if you have a 
%  thorn without the Cactus Flesh download this from the Cactus
%  homepage at www.cactuscode.org)
\usepackage{../../../../doc/latex/cactus}
\usepackage{xspace}
\newcommand{\grhayl}{\texttt{GRHayL}\xspace}
\newcommand{\glib}{\texttt{GRHayLib}\xspace}
\newcommand{\igm}{\texttt{IllinoisGRMHD}\xspace}
\newcommand{\hydrobase}{\texttt{HydroBase}\xspace}

\begin{document}

\title{GRHayLib}
\author{Samuel Cupp \\ Leonardo Rosa Werneck \\ Terrence Pierre Jacques \\ Zachariah Etienne}
\date{$ $Date$ $}

\maketitle

% Do not delete next line
% START CACTUS THORNGUIDE

\begin{abstract}
Provides access to the General Relativistic Hydrodynamics
Library (\grhayl). This library provides a suite of functions
for GRMHD evolution codes based on the \igm code.
\end{abstract}

\section{Introduction}

This thorn provides access to the General Relativistic
Hydrodynamics Library (\grhayl) through the GRHayLib.h
header file. The library contains all the core components
necessary to produce an \igm-like code in a modular,
infrastructure-agnostic form. These functions are tested
via unit tests using GitHub Actions in the
\href{https://github.com/GRHayL/GRHayL}{GRHayL repository}.

For comprehensive documentation of \grhayl's features, please
visit the \href{https://github.com/GRHayL/GRHayL/wiki}{GitHub wiki}.
As a brief overview, the library is split into several `gems'
which are entirely self-contained. These are Atmosphere, Con2Prim,
EOS, Flux\_Source, Induction, Neutrinos, and Reconstruction.
The \glib implementation provides access to all these gems in its
header file. Additionally, the thorn automatically initializes an
instance of the \texttt{ghl\_parameters} and \texttt{ghl\_eos\_parameters}
structs. The instances are named \texttt{ghl\_params} and \texttt{ghl\_eos},
and the pointers to these structs are provided by the GRHayLib.h
header.

\section{Equation of State}

One of the more complicated part of GRMHD simulations is properly
setting up the equation of state (EOS), especially for tabulated EOS.
\grhayl provides both hybrid and tabulated EOS, and this information
is passed to the library functions through \texttt{ghl\_eos}. \glib's
parameters control the initialization of this struct, so the user does
not need to set any of the struct's values unless they plan to manually
adjust these parameters during the simulation. The \texttt{EOS\_type}
parameter chooses the type of EOS, and the thorn then sets up the struct
using the other parameters associated with this choice.

For several variables, there are minimum, atmosphere, and maximum parameters.
We will refer to these as collectively limit parameters. For the limit
parameters a given EOS uses, the atmosphere parameter is required. The other
parameters either have defaults or are ignored if unset.

\subsection{Hybrid EOS}

The hybrid EOS requires several parameters to function properly.
First, it needs the number of elements in the polytropic approximation.
This is currently limited by \grhayl to 10 pieces. The transition values
of rho\_b are set by \texttt{rho\_ppoly\_in}, and the associated Gamma
values are stored in \texttt{Gamma\_ppoly\_in}. Note that for neos=1,
only a single Gamma is necessary. Additionally, the parameter \texttt{k\_ppoly0}
sets the polytropic constant for the first piece of the piecewise polytrope.
The rest of the constanst are computed automatically.

In addition to the polytropic setup above, the value of the thermal Gamma
\texttt{Gamma\_th} must be set.

The hybrid EOS only uses the rho\_b limit parameters. If unset, the
minimum is set to 0, and the maximum is set to 1e300, effectively turning
off both of these features.

\subsection{Tabulated EOS}

To properly configure for tabulated EOS, the most important parameter is
the path to the EOS table, given by the \texttt{EOS\_tablepath} parameter.
\glib automatically reads this table and makes it accessible to \grhayl
functions through \texttt{ghl\_eos}.

The tabulated EOS uses all the limit parameters (rho\_b, Y\_e, and T).
If unset, the mimima and maxima are set to table bounds.

\section{Conservative-to-Primitive Routines}

\grhayl provides a selection of con2prim routines along with automated
backup methods. \texttt{ghl\_params} passes the information from the
thorn parameters texttt{con2prim\_routine} and texttt{con2prim\_backup\_routines}
to the library's multi-method functions. Of course, any of the routines
can be called manually with logic dictated by the user. However, we
provide functions for automatically calling the selected routine,
as well as triggering the backup routines in the case of failure.
The ghl\_con2prim\_hybrid\_multi\_method() and
ghl\_con2prim\_tabulated\_multi\_method() functions implement this
feature, and the ghl\_con2prim\_hybrid\_select\_method() and
ghl\_con2prim\_tabulated\_select\_method() functions directly select
a single method based on an input value. \grhayl supports up to three
backups, but more complex behavior can be designed by the user using
the core con2prim functions. Since the individual con2prim routines
and the multi-method functions have identical argument lists, swapping
routines is straightforward and requires little work from the user.

These routines are iterative solvers, so they require an initial guess.
\igm used always used default guess, which is also provided in \grhayl.
The \texttt{calc\_primitive\_guess} parameter automatically sets the
guess to this default in the multi-method function, so set this to
"no" to use a user-defined guess. Manual use of the con2prim
routines require the user to either provide a guess or use the
ghl\_guess\_primitives() function.

% Do not delete next line
% END CACTUS THORNGUIDE

\end{document}
